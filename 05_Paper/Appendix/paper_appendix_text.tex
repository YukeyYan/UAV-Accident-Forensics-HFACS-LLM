% 简洁版论文附录
% 引导读者到GitHub查看详细信息

\appendix
\section{UAV Enhancement Patterns}
\label{appendix:uav_patterns}

This study identified 26 UAV-specific enhancement patterns extending the HFACS framework, organized across four hierarchical levels: L4 Organizational (6 patterns), L3 Supervision (5 patterns), L2 Preconditions (11 patterns), and L1 Unsafe Acts (4 patterns). The patterns were derived through literature review (56 papers, 2010-2025), ASRS incident analysis (847 reports), and expert validation (Cohen's $\kappa = 0.82$).

Complete pattern documentation, including definitions, sources, indicators, and HFACS mappings, is available at:

\url{https://github.com/YukeyYan/UAV-Accident-Forensics-HFACS-LLM}

\section{UAV Enhancement Patterns Details}
\label{appendix:patterns_details}

Tables~\ref{tab:uav_enhancement_patterns_part1} and~\ref{tab:uav_enhancement_patterns_part2} provide detailed documentation of the 26 UAV-specific enhancement patterns with representative case examples from ASRS incident reports.

% Include the first table (Organizational and Supervisory levels)
\begin{table*}[!htbp]
\centering
\caption{UAV-Specific Enhancement Patterns: Organizational and Supervisory Levels}
\label{tab:uav_enhancement_patterns_part1}
\begin{adjustbox}{width=\textwidth,center}
\begin{threeparttable}
\footnotesize
\begin{tabular}{@{}p{2.0cm}p{3.5cm}p{5.0cm}p{4.5cm}@{}}
\toprule
\textbf{HFACS Level} & \textbf{Enhancement Pattern} & \textbf{Definition} & \textbf{Representative Case Example} \\
\midrule

% Level 4: Organizational Influences
\rowcolor{blue!10}
\multicolumn{4}{c}{\textbf{Level 4: Organizational Influences (6 patterns)}} \\
\midrule

\textbf{Organizational Influences (L4)}
& \textbf{Part 107 Compliance}
& Organizational challenges in maintaining compliance with FAA Part 107 small UAS regulations
& Commercial operator failed to maintain current remote pilot certificates for 40\% of flight crew, resulting in unauthorized operations \\
\addlinespace[0.2ex]

& \textbf{LAANC Authorization}
& Organizational processes for Low Altitude Authorization and Notification Capability
& Delivery company's automated LAANC requests contained systematic altitude errors, causing 15 unauthorized airspace penetrations \\
\addlinespace[0.2ex]

& \textbf{VLOS Requirements}
& Organizational management of Visual Line of Sight operational requirements
& Survey company's policy allowed operations up to 2 miles without visual observers, violating VLOS requirements \\
\addlinespace[0.2ex]

& \textbf{BVLOS Operations}
& Organizational challenges in Beyond Visual Line of Sight operations
& Research institution conducted BVLOS flights without proper waiver, lacking required detect-and-avoid systems \\
\addlinespace[0.2ex]

& \textbf{Battery Constraints}
& Organizational resource management for power system limitations
& Fleet operator's inadequate battery replacement schedule led to 30\% capacity degradation and multiple forced landings \\
\addlinespace[0.2ex]

& \textbf{Energy Management}
& Organizational strategies for UAV energy and endurance management
& Emergency response team lacked standardized battery management protocols, causing mission failures during critical operations \\
\midrule

% Level 3: Unsafe Supervision
\rowcolor{green!10}
\multicolumn{4}{c}{\textbf{Level 3: Unsafe Supervision (5 patterns)}} \\
\midrule

\textbf{Unsafe Supervision (L3)}
& \textbf{GCS Interface Complexity}
& Supervisory challenges in managing Ground Control Station interface complexity
& Supervisor failed to ensure pilot training on new GCS software, leading to mode confusion during critical mission \\
\addlinespace[0.2ex]

& \textbf{Delayed Feedback Systems}
& Supervisory management of delayed feedback in remote operations
& Operations manager did not account for 200ms video latency in flight planning, causing multiple near-miss incidents \\
\addlinespace[0.2ex]

& \textbf{LAANC Integration}
& Supervisory oversight of LAANC system integration and usage
& Flight supervisor approved operations without verifying LAANC authorization status, resulting in controlled airspace violation \\
\addlinespace[0.2ex]

& \textbf{Airspace Authorization}
& Supervisory management of airspace authorization processes
& Chief pilot failed to establish procedures for real-time airspace status monitoring, causing TFR violations \\
\addlinespace[0.2ex]

& \textbf{Emergency Procedures}
& Supervisory oversight of UAV-specific emergency procedures
& Training supervisor did not require C2 link loss simulation, leaving pilots unprepared for actual communication failure \\

\bottomrule
\end{tabular}
\begin{tablenotes}
\footnotesize
\item All patterns derived from systematic analysis of 847 ASRS UAV incident reports (2016-2023).
\item Case examples are anonymized and representative of multiple similar incidents.
\end{tablenotes}
\end{threeparttable}
\end{adjustbox}
\end{table*}

% Include the second table (Preconditions and Unsafe Acts levels)
\begin{table*}[!htbp]
\centering
\caption{UAV-Specific Enhancement Patterns: Preconditions and Unsafe Acts Levels}
\label{tab:uav_enhancement_patterns_part2}
\begin{adjustbox}{width=\textwidth,center}
\begin{threeparttable}
\footnotesize
\begin{tabular}{@{}p{2.0cm}p{3.5cm}p{5.0cm}p{4.5cm}@{}}
\toprule
\textbf{HFACS Level} & \textbf{Enhancement Pattern} & \textbf{Definition} & \textbf{Representative Case Example} \\
\midrule

% Level 2: Preconditions for Unsafe Acts
\rowcolor{orange!10}
\multicolumn{4}{c}{\textbf{Level 2: Preconditions for Unsafe Acts (11 patterns)}} \\
\midrule

\textbf{Preconditions (L2)}
& \textbf{C2 Link Reliability}
& Command and Control link reliability as precondition for safe operations
& Pilot lost C2 link for 45 seconds due to interference, unable to regain control before UAV entered restricted airspace \\
\addlinespace[0.2ex]

& \textbf{Telemetry Accuracy}
& Accuracy and timeliness of telemetry data transmission
& GPS altitude readings were 50ft inaccurate due to multipath interference, causing terrain collision during low-altitude survey \\
\addlinespace[0.2ex]

& \textbf{Range Limitations}
& Communication and control range limitations affecting operations
& Pilot attempted operation at 3-mile range despite 2-mile equipment limitation, resulting in complete signal loss \\
\addlinespace[0.2ex]

& \textbf{Signal Degradation}
& Progressive degradation of communication signals during operations
& Gradual signal weakening over 10 minutes went unnoticed, leading to delayed response during emergency maneuver \\
\addlinespace[0.2ex]

& \textbf{Mode Confusion}
& Confusion between different flight modes and automation levels
& Pilot believed UAV was in manual mode but was actually in GPS hold, causing unexpected behavior during landing approach \\
\addlinespace[0.2ex]

& \textbf{Automation Dependency}
& Over-reliance on automated systems affecting manual skills
& When autopilot failed, pilot's degraded manual flying skills resulted in hard landing and structural damage \\
\addlinespace[0.2ex]

& \textbf{Weather Sensitivity}
& UAV sensitivity to weather conditions affecting operations
& 15-knot crosswind exceeded small UAV's capability, but pilot continued operation resulting in loss of control \\
\addlinespace[0.2ex]

& \textbf{Spatial Disorientation}
& Loss of spatial orientation due to remote operation characteristics
& Pilot lost orientation during FPV flight in featureless terrain, unable to determine UAV attitude or position \\
\addlinespace[0.2ex]

& \textbf{Visual Limitations}
& Limited visual references and environmental cues in remote operations
& Camera's limited field of view prevented detection of approaching aircraft until collision was unavoidable \\
\addlinespace[0.2ex]

& \textbf{Battery Limits}
& Battery capacity and performance limitations affecting operations
& Cold weather reduced battery performance by 40\%, causing unexpected power loss during return flight \\
\addlinespace[0.2ex]

& \textbf{Automation Degradation}
& Degradation of automated system performance over time
& Uncalibrated IMU caused gradual drift in autonomous flight path, leading to controlled airspace violation \\
\midrule

% Level 1: Unsafe Acts
\rowcolor{red!10}
\multicolumn{4}{c}{\textbf{Level 1: Unsafe Acts (4 patterns)}} \\
\midrule

\textbf{Unsafe Acts (L1)}
& \textbf{Autopilot Interactions}
& Errors in interaction with autopilot systems and mode management
& Pilot incorrectly programmed waypoint altitude as 400ft AGL instead of MSL, causing terrain collision \\
\addlinespace[0.2ex]

& \textbf{Flight Mode Switching}
& Errors during transitions between different flight modes
& Rapid switching between manual and GPS modes during emergency caused control oscillations and crash \\
\addlinespace[0.2ex]

& \textbf{Limited Visual References}
& Errors due to limited visual references in remote operations
& Pilot misjudged distance to obstacle due to camera perspective, resulting in collision with power line \\
\addlinespace[0.2ex]

& \textbf{Delayed Feedback}
& Errors caused by delayed feedback from remote systems
& 300ms video delay caused pilot to overcorrect during landing, resulting in hard impact and damage \\

\bottomrule
\end{tabular}
\begin{tablenotes}
\footnotesize
\item Patterns organized by HFACS hierarchical levels from organizational (L4) to individual acts (L1).
\item All case examples anonymized and derived from validated ASRS incident analysis.
\end{tablenotes}
\end{threeparttable}
\end{adjustbox}
\end{table*}

\section{Data and Code Availability}
\label{appendix:data_code}

All experimental data, code, and supplementary materials are available in the project repository referenced above.
